\documentclass{article}
\usepackage{url}
\input{/home/pd83/notation.tex}
\begin{document}
\title{Spherically-symmetric constant-density object}
\author{Peter C. Doerschuk}
\maketitle
Let $R_o$ (``o'' for ``object'') be the radius of the object and $\rho_0$
be the density of the object:
\begin{equation}
\rho(\vx)
=
\left\{
\begin{array}{ll}
\rho_0 & |\vx|\le R_o \\
0 & \mbox{otherwise}
\end{array}
\right.
.
\end{equation}
From Ref.~\cite[Eq.~16.127]{Jackson1975},
\begin{eqnarray}
\exp(i\vk^T \vx)
&=&
4\pi
\sum_{l=0}^\infty
i^l
j_l(kx)
\sum_{m=-1}^{+l}
Y_{l,m}^\ast(\thetap,\phip)
Y_{l,m}(\theta,\phi)
\\
\Leftrightarrow
\exp(i 2\pi \vk^T \vx)
&=&
4\pi
\sum_{l=0}^\infty
i^l
j_l(2\pi kx)
\sum_{m=-1}^{+l}
Y_{l,m}^\ast(\thetap,\phip)
Y_{l,m}(\theta,\phi)
\\
\Leftrightarrow
\exp(-i 2\pi \vk^T \vx)
&=&
4\pi
\sum_{l=0}^\infty
(-i)^l
j_l(2\pi kx)
\sum_{m=-1}^{+l}
Y_{l,m}(\thetap,\phip)
Y_{l,m}^\ast(\theta,\phi)
.
\end{eqnarray}
Therefore,
\begin{eqnarray}
P(\vk)
&=&
\int_{\vx\in\reals^3}
\rho(\vx) \exp(-i 2\pi\vk^T \vx) \dd\vx
\\
&=&
\int_{\vx\in\reals^3}
\rho(\vx)
\left[
4\pi
\sum_{l=0}^\infty
(-i)^l
j_l(2\pi kx)
\sum_{m=-1}^{+l}
Y_{l,m}(\thetap,\phip)
Y_{l,m}^\ast(\theta,\phi)
\right]
\dd\vx
\\
&=&
\int_{x=0}^{R_o}
\int
\rho_0
\left[
4\pi
\sum_{l=0}^\infty
(-i)^l
j_l(2\pi kx)
\sum_{m=-1}^{+l}
Y_{l,m}(\thetap,\phip)
Y_{l,m}^\ast(\theta,\phi)
\right]
x^2 \dd x
\dd\Omega
\\
&=&
\rho_0
4\pi
\sum_{l=0}^\infty
(-i)^l
\left[
\int_{x=0}^{R_o}
j_l(2\pi kx)
x^2 \dd x
\right]
\sum_{m=-1}^{+l}
Y_{l,m}(\thetap,\phip)
\left[
\int
Y_{l,m}(\theta,\phi)
\dd\Omega
\right]^\ast
\\
&=&
\rho_0
4\pi
\sum_{l=0}^\infty
(-i)^l
\left[
\int_{x=0}^{R_o}
j_l(2\pi kx)
x^2 \dd x
\right]
\sum_{m=-1}^{+l}
Y_{l,m}(\thetap,\phip)
\left[
\sqrt{4\pi}\delta_{l,0}\delta_{m,0}
\right]^\ast
\\
&=&
\rho_0
4\pi
(-i)^0
\left[
\int_{x=0}^{R_o}
j_0(2\pi kx)
x^2 \dd x
\right]
Y_{0,0}(\thetap,\phip)
\sqrt{4\pi}
\\
&=&
\rho_0
4\pi
\left[
\int_{x=0}^{R_o}
j_0(2\pi kx)
x^2 \dd x
\right]
\frac{1}{\sqrt{4\pi}}
\sqrt{4\pi}
\\
&=&
\rho_0
4\pi
\int_{x=0}^{R_o}
j_0(2\pi kx)
x^2 \dd x
.
\end{eqnarray}
\par
Replace $x$ by $\gamma$ defined by
\begin{eqnarray}
\gamma
&=&
2\pi kx
\\
\dd\gamma
&=&
2\pi k \dd x
\\
x=0
&\Leftrightarrow&
\gamma=0
\\
x=R_o
&\Leftrightarrow&
\gamma=2\pi k R_o
\end{eqnarray}
to get
\begin{eqnarray}
P(\vk)
&=&
\rho_0
4\pi
\int_{\gamma=0}^{2\pi kR_o}
j_0(\gamma)
\left(\frac{\gamma}{2\pi k}\right)^2
\frac{\dd\gamma}{2\pi k}
\\
&=&
\rho_0
4\pi
\frac{1}{(2\pi k)^3}
\int_{\gamma=0}^{2\pi kR_o}
\gamma^2
j_0(\gamma)
\dd\gamma
.
\end{eqnarray}
\par
Ref.~\cite[Eq.~10.47.3]{OlverLozierBoisvertClark2010} is
\begin{equation}
j_n(z)
=
\sqrt{\frac{\pi}{2z}} J_{n+\frac{1}{2}}(z)
.
\end{equation}
Therefore ($n=0$ and $z=\gamma$),
\begin{eqnarray}
P(\vk)
&=&
\rho_0
4\pi
\frac{1}{(2\pi k)^3}
\int_{\gamma=0}^{2\pi kR_o}
\gamma^2
\sqrt{\frac{\pi}{2\gamma}} J_{0+\frac{1}{2}}(\gamma)
\dd\gamma
\\
&=&
\rho_0
4\pi
\frac{1}{(2\pi k)^3}
\sqrt{\frac{\pi}{2}}
\int_{\gamma=0}^{2\pi kR_o}
\gamma^{(\frac{1}{2}+1)}
J_{\frac{1}{2}}(\gamma)
\dd\gamma
.
\end{eqnarray}
\par
Ref.~\cite[Eq.~10.22.1]{OlverLozierBoisvertClark2010} is the indefinite integral
\begin{equation}
\int z^{\nu+1} \calC_\nu(z) \dd z
=
z^{\nu+1} \calC_{\nu+1}(z)
.
\end{equation}
Therefore ($\calC=J$, $\nu=\frac{1}{2}$),
\begin{eqnarray}
P(\vk)
&=&
\rho_0
4\pi
\frac{1}{(2\pi k)^3}
\sqrt{\frac{\pi}{2}}
\left.
\gamma^{(\frac{1}{2}+1)}
J_{1+\frac{1}{2}}(\gamma)
\right|_{\gamma=0}^{2\pi kR_o}
\\
&=&
\rho_0
4\pi
\frac{1}{(2\pi k)^3}
\sqrt{\frac{\pi}{2}}
(2\pi kR_o)^{(\frac{1}{2}+1)}
J_{1+\frac{1}{2}}(2\pi kR_o)
.
\end{eqnarray}
\par
Ref.~\cite[Eq.~10.47.3]{OlverLozierBoisvertClark2010} in the reverse direction is
\begin{equation}
J_{n+\frac{1}{2}}(z)
=
\sqrt{\frac{2z}{\pi}}
j_n(z)
.
\end{equation}
Therefore ($n=1$ and $z=2\pi kR_o$),
\begin{eqnarray}
P(\vk)
&=&
\rho_0
4\pi
\frac{1}{(2\pi k)^3}
\sqrt{\frac{\pi}{2}}
(2\pi kR_o)^{(\frac{1}{2}+1)}
\sqrt{\frac{2\times 2\pi kR_o}{\pi}}
j_1(2\pi kR_o)
\\
&=&
\rho_0
4\pi
\frac{1}{(2\pi k)^3}
(2\pi kR_o)^2
j_1(2\pi kR_o)
\\
&=&
\rho_0
4\pi
R_o^3
\frac{1}{(2\pi kR_o)^3}
(2\pi kR_o)^2
j_1(2\pi kR_o)
\\
&=&
\rho_0
4\pi
R_o^3
\frac{1}{2\pi kR_o}
j_1(2\pi kR_o)
.
\end{eqnarray}
Except for the replacement of ``$k$'' by ``$2\pi k$'', this formula matches
Ref.~\cite[formula between Eqs.~20
  and~21]{ZhengDoerschukJohnsonBiophysicalJ1995}.
%\bibliographystyle{unsrt}
%\bibliographystyle{alpha}
\bibliographystyle{abbrv}
\bibliography{/home/pd83/references}
\end{document}
