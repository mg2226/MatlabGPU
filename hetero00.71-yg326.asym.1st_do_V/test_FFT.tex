\documentclass{article}
\newcommand{\dd }{{\rm d}}
\begin{document}
\title{DFT and Fourier Transform}
\author{Peter C. Doerschuk}
\maketitle
\section{Relationship between DFT and Fourier Transform}
DFT:
\begin{eqnarray}
X_d[k]
&=&
\sum_{n=0}^{N-1} x_d[n] \exp\left(-i \frac{2\pi}{N} nk \right)
\\
x_d[n]
&=&
\frac{1}{N} \sum_{k=0}^{N-1} X_d[k] \exp\left(i \frac{2\pi}{N} nk \right)
.
\end{eqnarray}
Fourier transform:
\begin{eqnarray}
X(f)
&=&
\int_{t=-\infty}^{+\infty} x(t) \exp\left(-i 2\pi ft\right) \dd t
\label{eq:Xf}
\\
x(t)
&=&
\int_{f=-\infty}^{+\infty} X(f) \exp\left(i 2\pi ft\right) \dd f
.
\end{eqnarray}
Suppose that $x(t)=0$ for $t<0$ and $t>T$.
Suppose that you want to approximate Eq.~\ref{eq:Xf} by an $N$-point
integration rule with equally-spaced abscissas ($=\delta=T/N$) and constant
weights ($=w=T/N=\delta$).
Let the approximate answer be denoted by $X_a(f)$.
Let $x_d[n]=x(n\delta)$.
Then
\begin{eqnarray}
X_a(f)
&=&
\sum_{n=0}^{N-1} x(n \delta) \exp\left(-i 2\pi f n \delta\right) \delta
\\
&=&
\delta
\sum_{n=0}^{N-1} x_d[n] \exp\left(-i 2\pi f n \frac{T}{N}\right)
\\
&=&
\delta
\sum_{n=0}^{N-1} x_d[n] \exp\left(-i \frac{2\pi}{N} (fT) n\right)
.
\end{eqnarray}
Suppose you focus on $f$ such that $fT=k\in\{0,\dots,N-1\}$.
Then
\begin{eqnarray}
X_a(f=k/T)
&=&
\delta
\sum_{n=0}^{N-1} x_d[n] \exp\left(-i \frac{2\pi}{N} k n\right)
\\
&=&
\delta
X_d[k]
.
\end{eqnarray}
\par
Since $T/N=\delta$, I can express $T$ in the form $T=N\delta$ and then
\begin{equation}
X_a\left(f=\frac{k}{N\delta}\right)
=
\delta
X_d[k]
\quad
(k\in\{0,\dots,N-1\})
.
\end{equation}
\section{Sinusoidal signals}
In continuous time,
\begin{eqnarray}
x(t)=\exp(i2\pi f_0 t)
&\leftrightarrow&
X(f)=\delta(f-f_0)
\label{eq:continuoustime:sinusoid}
.
\end{eqnarray}
In discrete time, for $k_0\in\{0,\dots,N-1\}$ let
\begin{equation}
x_d[n]
=
\exp\left(i\frac{2\pi}{N} n k_0\right)
\label{eq:discretetime:sinusoid}
.
\end{equation}
Then,
\begin{eqnarray}
X_d[k]
&=&
\sum_{n=0}^{N-1} x_d[n] \exp\left(-i \frac{2\pi}{N} nk \right)
\\
&=&
\sum_{n=0}^{N-1}
\exp\left(i\frac{2\pi}{N} n k_0\right)
\exp\left(-i \frac{2\pi}{N} nk \right)
\\
&=&
\sum_{n=0}^{N-1}
\exp\left(-i\frac{2\pi}{N} n (k-k_0)\right)
\label{eq:DFT:exp}
.
\end{eqnarray}
Suppose $k=k_0$.
Then
\begin{eqnarray}
X_d[k]
&=&
\sum_{n=0}^{N-1}
\exp\left(-i\frac{2\pi}{N} n 0\right)
\\
&=&
\sum_{n=0}^{N-1}
1
\\
&=&
N
.
\end{eqnarray}
Now suppose $k\neq k_0$.
The geometric sum is
\begin{equation}
\sum_{n=0}^{N-1}
\rho^n
=
\frac{1-\rho^N}{1-\rho}
.
\end{equation}
Eq.~\ref{eq:DFT:exp} is a geometric sum with
\begin{equation}
\rho
=
\exp\left(-i\frac{2\pi}{N} (k-k_0)\right)
\end{equation}
so that
\begin{eqnarray}
X_d[k]
&=&
\frac{
1-\exp\left(-i\frac{2\pi}{N} (k-k_0)\right)^N
}{
1-\exp\left(-i\frac{2\pi}{N} (k-k_0)\right)
}
\\
&=&
\frac{
1-\exp\left(-i 2\pi (k-k_0)\right)
}{
1-\exp\left(-i\frac{2\pi}{N} (k-k_0)\right)
}
\\
&=&
\frac{
1-1
}{
1-\exp\left(-i\frac{2\pi}{N} (k-k_0)\right)
}
\\
&=&
0
.
\end{eqnarray}
Therefore, for $k_0\in\{0,\dots,N-1\}$,
\begin{eqnarray}
x_d[n]
=
\exp\left(i\frac{2\pi}{N} n k_0\right)
&\leftrightarrow&
X_d[k]
=
N \delta_{k,k_0}
.
\end{eqnarray}
\par
Suppose you want to think of a continuous-time sinusoidal signal of the
form shown in Eq.~\ref{eq:continuoustime:sinusoid} that is sampled to give
$x_d[n]$ of the form shown in Eq.~\ref{eq:discretetime:sinusoid}, that is,
\begin{eqnarray}
x_d[n]
&=&
x(n\delta)
\\
&=&
\exp(i2\pi f_0 n\delta)
.
\end{eqnarray}
Therefore we need
\begin{equation}
f_0 \delta
=
\frac{k_0}{N}
.
\end{equation}
So the only continuous time signals that will work are those with
\begin{equation}
f_0
=
\frac{k_0}{N\delta}
\end{equation}
where $k_0\in\{0,\dots,N-1\}$.
\end{document}
